%% TODO:
%% adjust consistent style, variable name
%% remove the most colons (not necessary)

%% LaTeX-Beamer template for KIT design
%% by Erik Burger, Christian Hammer
%% title picture by Klaus Krogmann
%%
%% version 2.1
%%
%% mostly compatible to KIT corporate design v2.0
%% http://intranet.kit.edu/gestaltungsrichtlinien.php
%%
%% Problems, bugs and comments to
%% burger@kit.edu

\documentclass[18pt]{beamer}

%% SLIDE FORMAT

% use 'beamerthemekit' for standard 4:3 ratio
% for widescreen slides (16:9), use 'beamerthemekitwide'

\usepackage{templates/beamerthemekit}
% \usepackage{templates/beamerthemekitwide}
\usepackage[utf8]{inputenc}
\usepackage{multicol}

\usepackage{tikz}
\usepackage{amsmath}
\usepackage{verbatim}
\usetikzlibrary{arrows,shapes}
\usetikzlibrary{snakes}
\setbeamercolor{alerted text}{fg=red} 
%% TITLE PICTURE

% if a custom picture is to be used on the title page, copy it into the 'logos'
% directory, in the line below, replace 'mypicture' with the 
% filename (without extension) and uncomment the following line
% (picture proportions: 63 : 20 for standard, 169 : 40 for wide
% *.eps format if you use latex+dvips+ps2pdf, 
% *.jpg/*.png/*.pdf if you use pdflatex)

%\titleimage{mypicture}

%% TITLE LOGO

% for a custom logo on the front page, copy your file into the 'logos'
% directory, insert the filename in the line below and uncomment it

%\titlelogo{mylogo}

% (*.eps format if you use latex+dvips+ps2pdf,
% *.jpg/*.png/*.pdf if you use pdflatex)

%% TikZ INTEGRATION

% use these packages for PCM symbols and UML classes
% \usepackage{templates/tikzkit}
% \usepackage{templates/tikzuml}


% the presentation starts here

\title[Parallel Algorithm for Closest Pair Problem]{Parallel Algorithm for Closest Pair Problem}
\subtitle{}
\author{Ge Wu}

\institute{Institute for Theoretical Informatics}

% Bibliography

\usepackage[citestyle=authoryear,bibstyle=numeric,hyperref]{biblatex}
 
\addbibresource{templates/example.bib}
\bibhang1em

\begin{document}

% change the following line to "ngerman" for German style date and logos
\selectlanguage{english}

%title page
\begin{frame}
\titlepage
\end{frame}

%table of contents
%\begin{frame}{Gliederung}
%\tableofcontents
%\end{frame}

\section{Problem Description}

\begin{frame}{Problem Description}
\begin{block}{Closest Pair Problem}
\begin{itemize}
\item Given $n$ \textbf{different unordered} points $P = \{p_1,p_2, ... ,p_n\}$ in \textbf{unit square}: \\
$$p_i = (x_i, y_i) \in (0, 1) \times (0,1) \subset \mathbb{R}^2$$
\item Find a pair of points with the smallest distance between them.
\item Find any pair if there's a tie.
\end{itemize}
\end{block}
\end{frame}

\begin{frame}{Background}

\end{frame}

\tikzstyle{st1}=[fill=blue,circle,inner sep=1pt]
\tikzstyle{st2}=[fill=red,circle,inner sep=1pt]
\tikzstyle{st3}=[fill=green,circle,inner sep=1pt]

\begin{frame}{Algorithm}	
\begin{columns}
\column{0.45\textwidth}
	\begin{itemize}
		
			\item<alert@2-3> Sample \textcolor{black}{($\boldsymbol{n^{2/3}}$ Points from $P$)}
%				\hspace{1em}$\color{black}{S = \{s_1,...,s_{\alert{\boldsymbol{n^{2/3}}}}\} \subset P}$ \\
%				\hspace{1em}$\color{black}{\delta = min(dist(s_i, s_j))}$
			\item<alert@4> Partition
			\item<alert@5-> Compute
	\end{itemize}
\column{0.55\textwidth}

	\begin{tikzpicture}[remember picture, scale = 5]
	\draw [<->] (0,1.1) node (yaxis) [above] {$y$} |- (1.1,0) node (xaxis) [right] {$x$};	      
	\coordinate (c0) at (0,0);  
    \coordinate (c) at (1,1);
    
    \coordinate (la) at (-0.01, 0.148);
    \coordinate (la0) at (-0.01, 0);
    \coordinate (lam) at (-0.02, 0.074);
    
	\coordinate (cc22) at (0.2986,0.2986);
	\coordinate (cc23) at (0.2986,0.4479);
	\coordinate (cc24) at (0.2986,0.5972);
	\coordinate (cc25) at (0.2986,0.7465);
	\coordinate (cc32) at (0.4479,0.2986);
	\coordinate (cc33) at (0.4479,0.4479);
	\coordinate (cc34) at (0.4479,0.5972);
	\coordinate (cc35) at (0.4479,0.7465);
	\coordinate (cc42) at (0.5972,0.2986);
	\coordinate (cc43) at (0.5972,0.4479);
	\coordinate (cc44) at (0.5972,0.5972);
	\coordinate (cc45) at (0.5972,0.7465);
	\coordinate (cc52) at (0.7465,0.2986);
	\coordinate (cc53) at (0.7465,0.4479);
	\coordinate (cc54) at (0.7465,0.5972);
	\coordinate (cc55) at (0.7465,0.7465);
    
    
	\draw[dashed] (yaxis |- c) node[left] {$1$}
        -| (xaxis -| c) node[below] {$1$};
    \draw (0,0) node[left,below] {$0$};
        
\node[st1](p1) at (0.10802,0.20208){};
\node[st1](p2) at (0.517,0.45389){};
\node[st1](p3) at (0.14316,0.42791){};
\node[st1](p4) at (0.55937,0.96605){};
\node[st1](p5) at (0.15796,0.62006){};
\node[st1](p6) at (0.76668,0.59539){};
\node[st1](p7) at (0.84871,0.72016){};
\node[st1](p8) at (0.91682,0.3469){};
\node[st1](p9) at (0.98697,0.51699){};
\node[st1](p10) at (0.50513,0.55669){};
\node[st1](p11) at (0.27142,0.1565){};
\node[st1](p12) at (0.10075,0.56206){};
\node[st1](p13) at (0.50785,0.6948){};
\node[st1](p14) at (0.58561,0.42646){};
\node[st1](p15) at (0.76289,0.83627){};
\node[st1](p16) at (0.082963,0.73139){};
\node[st1](p17) at (0.6616,0.36003){};
\node[st1](p18) at (0.81698,0.25421){};
\node[st1](p19) at (0.17105,0.38639){};
\node[st1](p20) at (0.93856,0.77555){};
\node[st1](p21) at (0.59048,0.73427){};
\node[st1](p22) at (0.44063,0.43028){};
\node[st1](p23) at (0.94192,0.69375){};
\node[st1](p24) at (0.65591,0.94521){};
\node[st1](p25) at (0.45195,0.78423){};
\node[st1](p26) at (0.3397,0.70557){};
\node[st1](p27) at (0.53262,0.10933){};
	\end{tikzpicture}	
	
	\begin{tikzpicture}[remember picture, overlay]
	\node<2-3> at (p1) [st2] {};	
	\node<2-3> at (p2) [st2] {};	
	\node<2-3> at (p3) [st2] {};	
	\node<2-3> at (p4) [st2] {};	
	\node<2-3> at (p5) [st2] {};	
	\node<2-3> at (p6) [st2] {};	
	\node<2-3> at (p7) [st2] {};	
	\node<2-3> at (p8) [st2] {};	
	\node<2-3> at (p9) [st2] {};	

	\path <3> [draw,thick] (p6) -- node [right]{$\delta$}(p7);

	\draw<4-12>[step= 0.7466, gray, very thin, dashed, xshift =12.5pt, yshift = 6.5pt] (c0) grid (c);	
	\draw<4> [snake=brace, thick] (la0) -- (la);	
	\draw<4> (lam) node[left] {$\delta$};
	
	\draw<5-12>[thick] (cc33) rectangle (cc44);
	\node<5-12> at (p2) [st2] {};	
	\node<5-12> at (p10) [st2] {};		
	
	\draw<5>[thick] (cc24) rectangle (cc35);
	\node<5> at (p26) [st3] {};	

	\draw<6>[thick] (cc34) rectangle (cc45);	
	\node<6> at (p13) [st3] {};	
	\node<6> at (p21) [st3] {};		
	
	\draw<7>[thick] (cc44) rectangle (cc55);	
	
	\draw<8>[thick] (cc43) rectangle (cc54);		
	
	\draw<9>[thick] (cc42) rectangle (cc53);
	\node<9> at (p17) [st3] {};
	
	\draw<10>[thick] (cc32) rectangle (cc43);
	\node<10> at (p14) [st3] {};	

	\draw<11>[thick] (cc22) rectangle (cc33);
	\node<11> at (p22) [st3] {};
	
	\draw<12>[thick] (cc23) rectangle (cc34);
	
	\node<13> at (p3) [st2] {};
	\node<13> at (p19) [st2] {};
	\end{tikzpicture}
	
\end{columns}
\end{frame}

%\begin{frame}{Running Time}
%\begin{itemize}
%\item Sample \alert{O(n)}
%	\begin{itemize}
%		\item Computer all pair of distances: $O((n^{2/3})^2) = O(n^{frac{4}{3}})$
%		\item Divide \& Conquer: $O(n^{2/3}\log n^{2/3}) = O(n^{2/3}\log n) \subset O(n)$
%		\item Other approach?
%	\end{itemize}
%\item Partition \alert{O(n)}
%	\begin{itemize}
%		\item At most $n$ cells contain point
%		\item Map points to their cells, then hashing
%		\item Divide the coordinates by grid step and truncate to integer.
%	\end{itemize}
%\item Compute \alert{O(n)}
%$$O(\sum_{i \in K} f_i) \subset O(n)$$
%\end{itemize}
%\end{frame}

\begin{frame}{Sample}
%How to calculate the shortest distance between samples?
\begin{itemize}
		\item Computer all pair of distances: $O((n^{frac{2}{3}})^2) = O(n^{frac{4}{3}})$
		\item Divide \& Conquer: $O(n^{frac{2}{3}}\log n^{frac{2}{3}}) = O(n^{frac{2}{3}}\log n) \subset O(n)$ \\
		More samples possible: 
		\item Other approach?
\end{itemize}
\end{frame}

\begin{frame}{Partition}
	\begin{itemize}
		\item At most $n$ cells contain point
		\item Map points to their cells, then hashing
		\item Divide the coordinates by cell length and truncate them to integer.
		\item $O(n)$
	\end{itemize}
\end{frame}

\begin{frame}{Compute}


\end{frame}

\begin{frame}{Another Approach for Step 1}

\end{frame}

\begin{frame}{Parallelization}

\end{frame}

\begin{frame}{Proof}

\end{frame}
%
%\begin{frame}{Q \& A}
%\begin{center}
%\Large{\textbf{Thank You for Your Attention}}
%\end{center}
%\end{frame}

\appendix
\beginbackup


\nocite{major}
\nocite{dietzfelbinger1997reliable}
\nocite{fortune1978note}
\nocite{chap13}
\nocite{web}
\nocite{khuller1995simple}
\nocite{Bentley:1976:DMS:800113.803652}

\begin{frame}[allowframebreaks]{Reference}
\printbibliography
\end{frame}


\begin{frame}{Load Balancing in Step 3}
testc
\end{frame}

\begin{frame}{Implementation of Hashing}

\end{frame}


\backupend

\end{document}
